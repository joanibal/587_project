\documentclass{article}
\usepackage[utf8]{inputenc}

\title{EECS 587 Project Proposal: Parallel-in-time algorithms for system sensitivities}
\author{Josh Anibal}
\date{\today}

\usepackage{natbib}
\usepackage{graphicx}
\usepackage{pdflscape}
\usepackage{afterpage}
\usepackage{lscape}
\usepackage{amsmath, esint}
\usepackage{geometry}
\geometry{ margin=0.75in}
\usepackage[utf8]{inputenc}
\usepackage[english]{babel}

\usepackage{floatrow}

\usepackage[outdir=./]{epstopdf}
\usepackage[parfill]{parskip}
\usepackage{algorithm}
\usepackage{algorithmicx}
\usepackage{algpseudocode}
% \usepackage{verbatim}
\usepackage{mathtools}
\DeclarePairedDelimiter\ceil{\lceil}{\rceil}
\DeclarePairedDelimiter\floor{\lfloor}{\rfloor}

% \epstopdfsetup{outdir=./}
% Table float box with bottom caption, box width adjusted to content
\newfloatcommand{capbtabbox}{table}[][\FBwidth]
% \setlength\parindent{0pt}
\graphicspath{{../figures/}}

% \setlength{\parindent}{4em}
% \setlength{\parskip}{1em}
% \renewcommand{\baselinestretch}{2.0}
\usepackage{float}

\usepackage{booktabs}

\begin{document}
\label{Eqn:DE}
\maketitle



% Turn in a short report which includes the verification and a brief description of how
% you decomposed the matrix, how you did communication, and your timing result. Analyze whether
% the timing represents perfect speedup and scaling, and, if not, why is the program not achieving
% it (or, if you have superlinear scaling, why that occurred). Note that you have scaling in terms of
% the size of the matrix, as well as in the number of processors.

\section*{Proposed Project}
In design optimization the bottleneck is often computing the sensitivities because information must follow the path of the model in reverse in a serial fashion.
This is similar to backward propagation for neural network except a linear system may need to be solved for each step making some steps more expensive and the process must be repeated for each objective and constraint.
When propagating the information backward, the current step depends only on prior steps leading to a time-like decency structure between each backward propagation step.
It is for this reason I propose using algorithms developed by the parallel-in-time community to parallelize this process to speedup the computation of sensitivities.

\end{document}

